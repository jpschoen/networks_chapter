\documentclass[fleqn,12pt]{wlscirep}
% * <jpschoen@gmail.com> 2018-03-07T16:08:10.834Z:
%
% ^.


\linespread{1.5}





\title{\centering{Chapter 48 Outline\\
\vspace{1cm}
\large{Network-Analysis: Estimation and Inference}
\vspace{1cm}
}}
\author[1]{Bruce Desmarais}
\author[1]{John P. Schoeneman}
\affil[1]{Penn State, Political Science, University Park, Pond Lab}


%\keywords{Keyword1, Keyword2, Keyword3}

\begin{abstract}
In the first section we will give an overview of the most common models used for estimation in network analysis and the underlying inference framework for each model. In the second section we will discuss the advantages and disadvantages of each model to give practitioners guidance in selecting a model given their data and inference goals. In the last section we will conduct simulation studies with the goal of evaluating the effectiveness of out of sample predictions to assist practitioners selecting among these models.
\end{abstract}


\begin{document}



\flushbottom
\maketitle{}
% * <john.hammersley@gmail.com> 2015-02-09T12:07:31.197Z:
%
%  Click the title above to edit the author information and abstract
%
\thispagestyle{empty}


\section*{Models}

\begin{itemize}
\item LSM (Bilinear and Euclidean), mixed membership stochastic blockmodel, SOAM/Siena, ERGM (including TERGM and GERGM)

\end{itemize}

\subsection*{Inference Frameworks}

\begin{itemize}
\item ML vs Bayesian estimation 

\end{itemize}

\section*{Model Selection}

\begin{itemize}
\item Directed vs Undirected
\item Edge Types: Binary, Ordinal ,Count, Continuous
\item Parameters
\item Network vs Actor Inference
\item Longitudinal variations 
	\begin{itemize}
	\item Discrete time, Autoregressive
	\item Continuous time, process based
	\end{itemize}
\item Missingness
\end{itemize}



\section*{Simulation Study}

\begin{itemize}
\item 

\end{itemize}

\newpage

\bibliography{bibliography}







\end{document}